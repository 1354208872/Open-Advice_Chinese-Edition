\chapterwithauthor{Eugene Trounev}{Use of Color and Images in Design Practices}

\authorbio{An active member of Free Software and KDE for about 6 years, Eugene
Trounev started in KDEGames and followed through the entire KDE3-to-KDE4
transition. Nowadays he is mostly taking care of KDE's web presence and main
desktop appearance.}

\noindent{}Since the most ancient times people have used the power of images and color to
pass information, draw attention, and distract attention. The infamous
saying goes ``A picture is worth a thousand words'', and it could not be more to
the point. From the way we dress, to flashy neon light of downtown stores across
the globe -- every color, every shape and every curve has a purpose.

Knowing the purpose however is not that hard, since all of those variations of
hues and lines are put together to be read and felt by every one of us. It is
true therefore that a great design must come straight from the heart, as it is
supposed to speak to the heart in the first place. Nonetheless, just the heart
alone would not be able to make a great design, if some rules are not set and
followed at first.

\section*{Colors and textures}

There are many different ways to classify the colors into categories, but many of
them focus on physical or chemical properties of light or ink, and though they are
important in the end, those will not help you make an appealing design. The one way
that I found works best is to split colors into warm and cool. Simply speaking,
warm colors are those closer to the shade of red. They are: red, orange and
yellow. Cool colors, on the other end, are the ones running towards blue. They are: green,
blue and to a lesser extend violet. It is important to remember that cool is
also calm and breathy, while warm is impulsive and dangerous. So, depending on
what feelings you wish to awaken within your audience, you should use either
warmer or cooler colors. Draw attention with warm and inform with cool.
Overuse of either will result in either overheating -- creating negative feelings
in your viewer, or freezing-over -- causing indifference.

It is important to remember that black, white and grays are colors, too. These,
however, are neutral. They cause no feeling, but rather set an atmosphere. The
properties of these will be discussed later.

Every image is first and foremost a collection of colors, and as such will abide
by the rules of color management. Determining the dominant color of your image is
the key to success. Try to see the big picture, and do not concentrate on
details. A good way to do this is by setting an image against some dark
background, then taking a few steps back and observing it from a distance. Which
color do you see the most of?

Not all images have a dominant color, however. Sometimes you may come across
color bloat, where no matter how hard you look you can not determine which hue
dominates. Try to avoid such pictures, as they will inevitably confuse your
viewer. When confronted with imagery like that, people tend to look away quickly
and it will not give a good impression, no matter what it speaks of.

Beside color, pictures also have a texture, as ultimately they are nothing but a
collection of textured colors. Detecting the dominant texture of an image is not as
straight forward as its color, as textures are seldom obvious, especially in
photographs. There are however a few pointers to help you. Human nature causes
us to be drawn to curved, so called ``natural'' shapes, while angular,
sharp-looking shapes are considered less attractive. That is why an image of a
curved, green leaf would appeal to more people then that of a metal spike.

To summarize: the key to a successful, appealing design is a good, well
balanced combination between color and texture in the images used.

\section*{Texts and spaces}

An equally important aspect of any good design is the use of text and
spaces around it. And just like it is with the image textures and color, you
should always remember that people like to breathe. This means that there should be
sufficient space in and around the text to make it easier to spot, read and
understand.

Consider an example of two pages -- one coming from a romantic novel, while the
other is taken straight from a legal document. You would most likely prefer the
romantic novel over a legal document any day, but do you know why? The answer is
simply because you like to breathe. A page from any romantic novel is likely to
contain three important elements: a) conversations; b) paragraphs; c) extra wide
margins, while most legal documents normally contain neither. All of the
aforementioned elements make the page feel alive and dynamic, while the absence
of those make it look like a solid wall of text. Human eyes, being more
accustomed to a certain degree of variety of sights, feel more at ease when
presented with spacious, fluid layouts.

This does not however imply that every text must have all those three elements in
order to seem more attractive. Far from it. Any text can be made easy and
enjoyable by injecting enough air into the flow.

Air, or space, can come through a variety of ways, such as: letter, line and
paragraph spacing; content, section, and page margins; and finally letter size. Try
to keep at least one character-tall space between your paragraphs and lines, and
two character-tall space between sections in your text. Allow generous spacing
around the text on a page by setting your margins wide enough. Try to never go
below 10-points font size for your paragraph text, while keeping headings large
enough to stand out.

\section*{Attraction and information}

Just like animals, human beings are often attracted by bright splotches of color
and unusual texture, and the more captivating the sight is, the more oblivious
people become towards other potential points of interest. This simple rule of
attraction has been used since the most ancient times by females and males
alike to drive the attention of others away from certain things they did not want
to be noticed. The best example of such a trickery is the work of a street
magician, who often distracts viewers’ attention by use of smoke, flames or
flashy attire.

It is important to remember here that words are visual too, as they produce
specific associations and visions. The very same trick that can be done with
smoke and fires can also be achieved through creative use of wording. By far the
best example of a trickery done with words is our every day price tags. Ever
wondered why retailers love those .99s and .95s so much? That is because \$9.95, or
even \$9.99 looks more attractive than \$10.00, even though in reality they
have the same impact on your wallet. Trow an ``old'' \$10.00 price tag noticeably
crossed through with a thick red line into the mix and you got yourself a great
customer magnet.

\section*{Conclusion}

Great, attractive design is achieved by following these simple rules: a) choose
your imagery wisely; b) make good use of colors and textures to create an
atmosphere; c) give your viewer some room to breathe; d) draw the attention
away from the parts that matter the least, and towards those that matter the
most.

This short essay is not meant to cover the whole wide spectrum of various design
styles, techniques and rules, but rather to give you -- the reader -- a starting
point you could carry on building upon.
