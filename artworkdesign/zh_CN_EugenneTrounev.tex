\chapterwithauthor{Eugene Trounev}{在设计练习中使用多媒体元素}

\authorbio{一个活跃了六年的开源软件和KDE的成员,Eugene Trounev开始致力于KDEGames和参与整个KDE3到KDE4的转换。如今,他较多得关注KDE的web表现和主要桌面界面}

\noindent{}自从大多数之前些的人们使用的图片和色彩去传递信息,还有吸引和转移注意力起,就有这么一句不著名的话:“一图胜万语”,再也没有一个一句话能更好解释这个观点了。通过这个方法我们描绘着,闹市区存储的闪亮氖光灯每一种颜色,每一种色彩,每一个曲线都是有意义的。

知道了这个作用,当然不只是难,因为很多种可变的颜色与曲线的变化交汇在一起让其变得可以给所有的人可读和感受。一个好的设计确实一定要来自于内心,反应与来自你心底的第一个反应。虽然如此,当你觉得孤单的时候是做不出一个好的设计,或者如果一开始一些规则没有预设好或者没有遵守。

\section*{色彩和纹路}

有很多不同的方法去分类色彩去放入分类集合里,但是很多人只关注物理或者化学的墨水或者灯光的道具,但是往往这些都不是最重要的,这些将不能帮助你实现一个有魅力的设计。我发现有一个方法可以最好得让冷暖颜色分隔。简单来讲,暖色更接近红色的感觉。有红,橙,黄。冷色在另外一头。有绿,蓝,还有更接近到紫罗兰。这是非常重要的去记住冷色同时是冷静和沉思的感觉,而暖色调则是冲动和危险的感觉。所以,根据你要表达给观看者的感觉去使用冷色或者暖色。吸引注意力通过暖色和冷色的表达。过度使用其中的一者将会导致过头,会导致一种消极的影响带给你的观看者,是狂热或者冰冷,这是两种完全不同的感受。

记住黑,白,灰也是彩色的是非常重要的。当然他们是非彩色的,但是这些颜色可以设置一种气氛。这是他们的一种功能,我稍后讨论。

每一个图像都是重要的和首位的一种颜色的聚集,这种将留下通过颜色规则的管理。相信图片主色调是你成功的关键。尽量去看大图,而不要关注细枝末节。一个好的方法就是设背景为暗色掉,然后退几步远看看这个色调是不是你想要的。

一个图片不应全是主色调。当然一些时候你会突然发现那些很难让你有信心的主控颜色是多么得不和谐感觉肿胀。尽量避免这种图片,然后这些图片也是不可避免地出现在你的观看着们的眼里。当面对这种情况的时候,人们更像是会快速看过而不会留下好的印象,而不管图片诉说的是什么东西。

除了颜色图片,当然还有版式,作为最重要的一部分,它不仅仅只是一种粗糙质地的颜色的集合。检测一下图片的主控纹路是不是直接表达他的颜色,纹路应该更少得显现,尤其是在照片里。当然也有几点可以帮助你。人的自然天性会审美疲劳导致实物弯曲,所以称之自然的“形状”。当一个矩形,锋利的外表只会有更少的吸引了。这就是为什么很多图片是弯曲的,绿叶更接近自然,而与之相对的就是金属钉了。

总结一下:要成功的话,显现的设计是要优秀的,在图片中使用颜色与纹路要做好平衡。

\section*{文本 和 空间}

一个在任何好的设计中等同重要的方面是文本与其周围空间的使用。就想图片的纹路和颜色一样,你应该始终牢记人们总是喜欢静下心阅读。这就意味着这应该有充足的空间在文本的四周去让文本更方便得去阅读,去发现,去理解。

看两个例子 -- 一个来自一个浪漫的小说,而另外一篇则是直接来自法律文档。有将会更喜欢那个浪漫的小说而不是那个那个文档,但是你知道为什么么?一个非常简单的答案包含三个非常重要的基础:1.交互 2.分段 3. 额外的边缘,当然大多数的法律文档一般都不包括。所有如上所述的原理使得小说变得更加灵活,而不是那种一看就是很死板的文章。人的眼睛更加得习惯于那种有大量明确段落的文章,而排斥那些一看就不着边际的排版。

让所有的文件都是要遵循着三条原则肯定可以让这个文章更有吸引力。远非如此,只要一个文本可以通过在字里行间添加空间可以使它更易阅读从而让读者享受阅读。

空间或者间距的使用的技巧,可以在很多途径中使用,比如信,行于段落的间隙。标题,章节,还有每页的每个角落。还有信纸的大小。尽量去保持段落与行之间有一个字符的宽度。大方允许文本周围有一般空间大小的剩余通过设置页面足够宽。尽量去不要用小于 10-points 的字体大小去设置你的文本大小,要保持标题足够大以至于其能突显。

\section*{吸引力 和 信息}

就好像动物,人类通常会被花俏的色彩和多变的纹路所吸引,而且越是捕捉这种缤纷,越来越多的人变得对其他潜藏的有趣点所吸引。这个简单的准则很早就已经被应用在男性和女性同样去从其他人身上转移那些他们不想被注意的事。一个很好的例子就是那些花俏的杂志使用烟,打火机或者闪耀的服装在其封面去吸引读者的注意。

重要的是记住文字也是真实的,它产生了一种特殊的契合和想象,这非常不同于单纯是烟和火达不能达到的那种只有文字才能创造的想象。最好的例子就是用文字写出的标价小花招。那些想要的商品的标价都是不整的。如9.99元,这远比10远更具有吸引力。在现实生活中,这作用同样体现在我们使用钱包的时候。当从一堆钱取出我们拿出一张旧的100元,会更能吸引人的注意。

\section*{结论}

好的,有吸引力的设计是要遵循一下几个原则:1.选择一个好的想象构思 2.使用好的颜色和纹路去创造一个好的环境气氛 3.给你的阅读者一些空间去喘息 d.把吸引力从那些不重要的吸引到重要的上面来。

