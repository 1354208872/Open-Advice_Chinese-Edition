\chapterwithauthor{Máirín Duffy Strode}{别害羞}

\authorbio{Máirín Duffy Strode高中时就开始使用自由与开源软件,
并且在过去的8年中一直是名贡献者。她涉足Fedora和GNOME两个社区,
并且在交互设计、品牌化,以及诸如Spacewalk, Anaconda, virt-manager, 
SELinux和SSSD等开源软件应用的图标设计方面有突出贡献。同时,
她使用诸如GIMP和Inkscape这样的开源软件工具,不遗余力地教授
孩子们设计技巧。她也是这些工具的极力推崇者。她是Fedora设计
团队的领导,也是Red Hat公司的一名高级交互设计师。}

\noindent{}在成为一名贡献者之前,我了解并使用自由和开源软件有很长一段时间了。
这并非缺少尝试--有很多错误的开始,而我屈服于他们主要是因为太害羞并且不敢完成它们。
出于这些错误的开始,也出于在自由与开源软件的项目中的培训,我有5条建议,
可以让作为设计师的你努力提升为自由与开源软件项目的贡献者:

\section*{1. 你是我们需要的和想要的人 (很重要!)}
我第一个错误的开始是,当我在Rensselaer Polytechnic Institute读计算机专业一年级的
时候发生的。有一个项目是我经常使用的,我想参与到其中。我对于项目中的人一个也不认识
(也不认识任何参与自由软件的人),因此我很尴尬的参与着。这个项目的网站上提示说他们
需要帮助,他们有一个聊天频道,于是我在那里“潜伏”了一两周。一天,在谈话的间隙我发言了:
“我说过我是一个对实用性感兴趣的计算机专业学生,我愿意加入。”

“走开!”某人回复道。我同时被告知,\emph{我的}帮助既不是他们需要的,也不是他们想要的。

这使我好几年不去参加开源项目--仅仅是聊天频道的几句粗话,使得我几乎5年内不敢再次尝试。
直到很后来的时候我才发现,在聊天频道里那个驱使我不参与那个项目的人,是处于该项目边缘的,
并且那样的行为已经长达多年了,而实际上我并没做错什么。若我坚持尝试参与并与其他人交谈,
我那时便可以和开始动工了。

若你想要对自由和开源软件做贡献,我向你保证这里有一个项目真的需要你的帮助,尤其适合于
有设计思想的人!你对web设计感兴趣吗?图标设计?适用性?皮肤?图形界面模型?我对很对
FOSS开发者讲过,这些人不仅急需要这种帮助,也深深地感激,喜爱你提供的一点一滴。

若你开始参与一个项目并且遇到了反对,从我的经验中学习吧,不要立即放弃。要是那个项目
表明它不适合你,不必担心,请继续你的脚步。机会就是,去寻找一个你喜欢的并且它也喜欢
你的项目。

\section*{2. 帮助一个项目就是帮助自己和别人}
如今,许多自由和开源软件由程序员和工程师把持,尽管一些算是幸运,有一两个有创造力
的人参与,但对于大多数项目而言,一个设计师、艺术家或其他有创造力的出现是可望而不可及的。
换句话说,尽管他们知道他们需要你的技能,他们可能并不知道他们能向你寻求什么样的
帮助,也不知道他们应当提供给你什么样的信息以使你多产,甚至是如何与你高效工作的
基本事情。

当我开始参与到各种FOSS项目中时,我遇到了许多先前从未直接和设计师工作过的开发者。
一开始,我感觉自己没什么用处。在聊天频道我无法参与他们的讨论,因为他们谈及了我
不了解的后端技术细节。当他们费心地关心我时,会问一些诸如“我在这里该用什么颜色?”
或“我该用哪种字体?”等问题。作为交互设计师,我真心想要的是参与决策--关于如何
实现项目需求。如果一个用户需要一个特定的功能,我想对它的设计说些话--但我并不知道
这些决定是在什么时候、什么地点被做出的,感觉自己被蒙在鼓里。

设计需要很多技能(说明,排版,交互设计,是就饿设计,图标设计,图形设计,辞藻等),
而任何给定的设计师几乎不具备这些的全部。因而可以被理解的是,开发者不确定要问什么。
并非他们要把你蒙在鼓里--他们只是不清楚你如何需要或者想要参与到其中。

帮助他们就是帮助你。要通过提供你做过的其它作品的样例,让他们明确你愿意提供的
作品类型。让他们了解你的需求,他们才能更好地理解如何让你参与到他们的项目中。
例如--当你第一次参与到一个项目的开端时,花时间概括它的设计流程,并且把它放到
主要开发列表上,以便其它贡献者可以跟进。如果你需要在此过过程中指定地方“输入”,
在你的概要图上标记出来。如果你不确定特定的事情怎样发生--例如开发一个新的特性
的过程--找旁边的人问一下,让他们帮你走一下流程。如果某人要求你做超乎你技术能力
的事情--例如使用版本控制工作--而你对此感觉不舒服,讲出来吧。

讨论你的处理与需求,可以避免让项目被迫做猜测。反过来,他们可以将你的天分发挥得
淋漓尽致。

\section*{3. 提问. 多多益善. 没有愚蠢的问题.}
我们注意到有时候在Fedora社区,当新的设计人员来到论坛时,他们不敢提问技术问题,
因为怕看上去很愚蠢。

实际上,开发人员可以如此的细化,以至于有太多的技术细节超乎他们即时的专业知识,
他们甚至也都不理解--甚至同一项目中也会发生此种情况。区别在于,他们不怕提问--
因此你更不该怕提问!例如在我的交互设计工作方面,我一定要让项目中众多的人理解
软件中的特定工作流程是怎样的,因为这在许多子系统中是通用的,而并非每个项目中
的人都理解子系统如何工作。

如果你不确定做什么,或者你不确定怎样开始,或者你不确定为什么某个聊天中某人的
谈哈那么搞笑--提问吧。很大可能上某人会告诉你他们也互相不认识,而不会把你看作
愚蠢的。大多数情形下,你会学习到一些新的东西,它们能帮你成为更好的贡献者。

尤其有效的方法是,找一个指导人--一些项目甚至有指导性的程序--如果他们不介意在
你有问题的时候提供你答案做指导,那就问吧。

\section*{4. 分享并经常分享. 即便还没准备好. 尤其是还没准备好.}
我们在Fedora和其它一些自由与开源软件项目中发现了一些设计师,他们展示自己作品时
有点害羞。我知道,你们不想因为把自己作品拿出来展示而毁了名声,而且这些作品不是
你最好的或还没有完成。但是,自由与开源软件项目中一大部分就是经常分享,以及开放。

你越迟开始分享,别人越难为你提供可行的回馈,以及加入参与的机会。同时也使得别人
难以在你的部分进行合作,难以让人感觉到一种拥有、支持感,难以通过实现来维护。
在一些自由和开源项目中,不展示你即将实现的草图、设计和想法,甚至被看作是冒犯的!

在网上展示你的想法,模型或者设计,而不是用邮件的方式,这样一来项目中的人们可以
通过复制和粘贴URL的方式来参考你的设计资料--这在讨论时尤其方便。你的设计资料越
容易被找到,就越容易被使用。

尝试这条建议并且保持开放心态。早写分享你的工作,并时常分享之,并使你的源文件
可用。你会因所发生的一切而开心与吃惊的!

\section*{5. 在项目社区尽可能让自己可见。}
在我开始作为FOSS的一名贡献者的开始阶段,有一个工具完全不是有意地,但却实际上是
没有

博客作为一个工具,在我作为FOSS贡献者之一的开始阶段,后来发现无意中极大地帮助了我。
我开始时写博客只是为了自己,作为我在做的工作的“公文包”。我的博客对我而言是一笔
巨大财富,因为:
\begin{itemize}
 \item作为项目决策的历史记录,这是查找曾经的的设计觉得便捷方式--例如,弄明白当初为什么我们
 要放弃那个屏幕,或者为什么一个特定的方法尝试了却没成功。
 \item 作为交流工具,它帮助其他贡献者讨论你的项目,甚至用户可以了解到哪些工作正在进行中,
 了解到项目中会有即将到来的什么变化。许多次我忘记一个设计中的重要部分,此时这些人会即刻
 评论以让我知道!
 \item 能帮助我建立作为一名FOSS贡献者的名声,这使我在设计决策方面获得的信任与日俱增。
\end{itemize}

你写博客吗?找出那些聚合了你所在项目的成员的在读博客,并请求让他们添加你的博客
(在工具栏通常有一个一个链接可以这样做。)例如,你想要成为Fedora社区的一部分,
最大的博客聚合网站叫做Planet Fedora\footnote{\url{http://planet.fedoraproject.org}}. 
一旦加入了,就请写一篇博客介绍你自己,让人们了解你--提示\#1中所说的各方面的信息。

项目会很明确地有一个邮件列表或者论坛,那里会有讨论。加入其中,并发一份简介。
当你建立了项目资料--无论多小,无论完成度多低--写博客记录之,把他们上传到项目
wiki,谈及他们,并把链接发送到社区中聊天频道的杰出成员那里,以得到反馈。

让你的世界可视化,人们会与因你的工作而联系你,提供给你很酷的项目,以及建立在这
基础之上的其它机会。

以上,就是我希望作为一名设计师,在开始参与到自由和开源软件中时要知道的所有了。
如果这里有什么事你应该从中获得的,那就是不要害羞--请讲话,请让你的需求被人们
知道,请让别人知道你的才能,这样他们就可以让你在自由软件中找到自己的天地。
