\chapterwithauthor{Till Adam}{Free Software in Public Administrations}

\authorbio{Originally from a liberal arts and music background, Till Adam has spent the 
从最开始的自由艺术和音乐背景,Till Adam在差不多过去的十年都做软件。
last decade or so in software. He works at KDAB where he directs services, including
他在KDAB工作,包括为公司的一些有关自由软件的活动直接提供服务。
the company's Free Software activities. Till also serves on the board of directors
Till也服务于Kolab Systems AG的董事会,
of Kolab Systems AG, a company with a pure Free Software business model.
这是一个纯粹的以自由软件为商业模式的公司。
He lives with his wife and daughter in Berlin.}
他与他的妻子和女儿住在柏林。
\section*{Introduction}

Like, I imagine, many of the other authors in this collection of essays I
我想,像在这个文章的中收集的许多其他作者一样,
started contributing to Free Software when I was a student. I had decided
当我还是个学生时,就开始为自由软件做贡献。
relatively late in life to pursue a degree in Computer Science (having failed
我决定在相对较晚的年纪去追求一个计算机科学的学位
to become rich and famous as a musician) and was expecting to be quite
(本想成为一个荣华富贵的音乐家结果失败了)
a bit older than my peers when I graduated. So I thought it would be good
而当我毕业时比我的同龄人年纪更大,这我是相当期待的。
to teach myself programming, which I was not getting much of at
所以我认为自学那些我没有从学校得到很多的编程是很棒的,
school, to become more attractive to future employers, despite my age.
尽管我的年龄会很大,但成为未来的雇主时会更有吸引力,
After some forays into various smaller communities I eventually found my way
在经过一些进军各种小社区之后,
into KDE and started working on the email application. Thanks to
我最终发现我的方式可以应用到KDE并且可以在电子邮件应用程序中开始工作。
the extremely helpful and technically brilliant group of people I met there I
多亏了那些我遇见的给了我技术上的极大帮助的一群人,
was able to learn quickly and contribute meaningfully to the code base, getting
使我能够快速学习并贡献有意义的代码库,
sucked more and more into both the social circle and the fascinating technical
并且开始越来越多的卷入社交圈和那迷人的个人信息管理上的技术问题圈。
problem space of personal information management.

When KDAB, a company full of KDE people, asked me whether I wanted to help out
with some commercial work that was being done, as a student job, I was of
course thrilled to be able to combine making a living with my hobby of hacking
on KDE software. Over the years I then witnessed the adoption of KDE's personal
information management frameworks and applications by the public sector,
particularly in Germany, first hand and saw KDAB's business in this area grow.
As I transitioned into more coordinative roles it eventually became part of my
job to effectively sell and deliver services based on Free Software including KDE's products
to large organizations, particularly in the public sector.

It should be noted that much of the project work this text reflects upon was done in
cooperation with other Free Software businesses, namely g10code, the
maintainers of GNUPG and cryptography specialists, and Intevation, a consultancy focused
entirely on Free Software and its strategic challenges and opportunities. Especially
Bernhard Reiter, one of Intevation's founders, was instrumental to the selling and
running of many of these projects and whatever morsels of wisdom this text might contain
are likely products of his analysis and my many conversations with him over the years.

So if Bernhard and I could travel back in time and share insights with our younger, more
naïve selves, what would those insights be? Well, it turns out they all start with the
letter 'P'.

\section*{People}

As things stand today it is still harder for IT operations people and decision
makers to use Free Software than it is to use proprietary alternatives. Even in
Germany, where Free Software has relatively strong political backing, it is
easier and safer to suggest the use of something that is perceived as ``industry
standard'' or ``what everyone else does''; proprietary solutions, in other words.
Someone who proposes a Free Software solution will likely face opposition by
less adventurous (or more short-sighted) colleagues, close scrutiny by
superiors, higher expectations with respect to the results and unrealistic
budget pressure. It thus requires a special breed of person willing to take
personal risks, go out on a limb, potentially jeopardize career progress and
fight an uphill battle. This is of course true in any organization, but in a
public administration special persistence is required because things move
generally more slowly and an inflexible organizational hierarchy and limited
career options amplify the issue.

Without an ally on the inside it can be prohibitively difficult to get
Free Software options seriously considered. If there is such a person, it is important
to support them in their internal struggles as much as possible. This
means providing them with timely, reliable and verifiable information about
what goes on in the community the organization intends to interface with,
including enough detail to provide a full picture but reducing the
complexity of the communication and planing chaos that is part of the Free
Software way of working, at times, such that it becomes more manageable and
less threatening. Honesty and reliability help to build strong
relationships with these key people, the basis of longer term success. As
their interface to the wondrous and somewhat frightening world of Free
Software communities they rely on you to find the paths that will carry
them and their organization to their goals and they make decisions largely
based on personal trusts. That trust has to be earned and maintained.

In order to achieve this, it is important to focus not only on achieving
the technical results of projects, but also keep in mind the broader personal and organizational
goals of those one is dealing with. Success or failure of the current
project might not depend on whether an agency's project manager can show off
only marginally related functionality to superiors at seemingly random points
in the schedule, but whether the next project happens or not might. When you have
few friends, helping them be successful is a good investment.

\section*{Priorities}

As technologists, Free Software people tend to focus on the things that are
new, exciting and seemingly important at a technology level. Consequently we
put less emphasis on things that are more important in the context of an (often
large) public administration. But consider someone wanting to roll out a
set of technologies in an organization that intends to stick with it for a long
time. Since disruptive change is difficult and expensive, it is far more
important to have documentation of the things that will not work, so they can
be avoided or worked around, than it is to know that some future version will
behave much better. It is unlikely that that new version will ever be
practically available to the users currently under consideration, and it is far
easier to deal with known issues pro-actively than to be forced to react to
surprises.  Today's documented bug is, ironically, often preferable to
tomorrow's fix with unforeseeable side effects.

In a large organization that uses software for a long time, the cost of acquiring
the software, be it via licenses or as part of contracted custom development of
Free Software, pales in comparison to the cost of maintaining and supporting it.
This leads to the thinking that fewer, more stable features, which cause less load
on the support organization and are more reliable and less maintenance intensive
are better than new, complex and likely less mature bells and whistles.

While both of these sentiments run counter to the instincts of Free Software
developers, it is these same aspects that make it very attractive for the
public sector to contract the development of Free Software, rather than
spending the money on licenses for off-the-shelf products. Starting from a
large pool of freely available software, the organization can invest the
budgets it has into maturing exactly those parts that are relevant for its own
operations. It thus does not have to pay (via license costs) for the development of
market driven, fancy features it will not need. By submitting all of that work
back upstream into the community, the longer term maintenance of these
improvements and of the base software is shared amongst many. Additionally,
because all of the improvements become publicly available, other
organizations with similar needs can benefit from them with no
additional cost, thus maximizing the impact of tax payer money,
something any public administration is (or should be) keen to do.

\section*{Procurement}

So, if it is so clearly better use of IT budgets for government agencies to invest
into the improvement of Free Software and into the tailoring of it to its needs, why is it
so rarely done? Feature parity for many of the most important kinds of software has
long been reached, usability is comparable, robustness and total cost of ownership
as well. Mindshare and knowledge are of course still problems, but the key practical obstacle
for procurement of Free Software services lies in the legal and administrative
conditions under which it must happen. Changing these conditions requires work
on a political and lobby level. In the context of an individual project it is
rarely possible. Thankfully organizations like the Free Software Foundation Europe and
its sister organization in the US are lobbying on our behalf and slowly effecting
change. Let's look at two central, structural problems.

\paragraph*{Licenses, not Services}

Many IT budgets are structured such that part of the money is set aside
for the purchase of new software or the continued payment for the use of software
in the form of licenses. Since it was unimaginable to those who structured these
budgets that software could ever be anything but a purchasable good, represented
by a proprietary license, it is often difficult or impossible for the IT decision
makers to spend that same money on services. Managerial accounting will simply not hear of it.
This can lead to the unhappy situation that an organization has the will and the
money to improve a piece of Free Software to exactly suit its needs, deploy and run
it for years and contribute the changes back to community, yet the plan can not
go forward unless the whole affair is wrapped in an artificial and unnecessary sale
and purchase of an imaginary product based on the Free Software license.

\paragraph*{Legal Traps}

Contractual frameworks for software providers often assume
that whoever signs up to provide the software fully controls all of the involved 
copyrights, trademarks and patents. The buying organization expects to be indemnified against various
risks by the provider. In the case of a company or an individual providing a solution
or service based on Free Software that is often impossible since there are other
rights holders that can not reasonably be involved in the contractual arrangement.
This problem appears most pointedly in the context of software patents. It is practically
impossible for a service provider to insure against patent litigation risks which makes
it very risky to take on the full responsibility.

\section*{Price}

Historically, the key selling point of Free Software that has been communicated
to the wider public has been its potential to save money.
Free Software has indeed made large scale cost
saving possible in many organizations and for many years now. 
The GNU/Linux operating system has spearheaded this development.
Because of its free availability for download was perceived in stark contrast 
to the expensive licenses of its main competitor, Microsoft Windows.
For something as
widely used and useful as an operating system, the structural cost benefit of
development cost put onto many shoulders is undeniable.  Unfortunately the
expectation that this holds true for all Free Software products has led to the
unrealistic view that using it will always, immediately and greatly reduce
cost. In our experience, this is not true. As we have seen in earlier sections
it does make a lot of sense to get more out of the money spent using Free
Software and it is likely that over time and across multiple organizations
money can be saved, but for the individual agency looking to deploy a piece of
Free Software there will be an upfront investment and cost associated with
getting it to the point of maturity and robustness required.

While this seems entirely reasonable to IT operations
professionals it is often harder to convince their superiors with budget power
of this truth. Especially when potential cost saving has been used as an
argument to get Free Software in the door initially it can prove very
challenging to effectively manage expectations down the road. The earlier the
true cost and nature of the investment is made transparent to decision makers,
the more likely they are to commit to it for the long haul. 
High value for money is still attractive and a software services provider that will
not continue to be available because the high price pressure does not yield
sufficient economic success is as unattractive in Free Software as it is in
proprietary license based business models. It is thus also in the interest of the
customers that cost estimations are realistic and the economic conditions of the
work being done are sustainable.

\section*{Conclusion}

Our experience shows that it is possible to convince organizations in
the public sector to spend money on Free Software based services. It is
an attractive proposition that provides good value and makes political
sense. Unfortunately structural barriers still exist, but with the help
of pioneers in the public sector they can be worked around. Given
sufficient support by us all, those working for Free Software on a political
level will eventually overcome them. Honest and clear
communication of the technical and economic realities can foster
effective partnerships that yield benefits for the Free Software community,
the public administrations using the software and those providing them
with the necessary services in an economically viable, sustainable way.

