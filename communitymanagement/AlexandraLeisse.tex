\chapterwithauthor{Alexandra Leisse}{庆幸我还有不知道的事物}

\authorbio{ Alexandra Leisse  创造了一个准许他人参与的平台,并且把她对软件和网络热情转变成一项职业.在经历了12个月的做软件和歌剧自由职业者过渡时期,又在KDE社区活动上花费不计其数的小时后,她加入诺基亚,以网上社区管理员的身份参加了Qt开发者网络.

\newline
她就是 Qt开发者网络和 Qt网上社区活动背后的那个女人,若不是为了获得歌剧表演的学位 ,她几乎不在公开场合唱歌.

\section*{入门}
当 Lydia 邀请我加入她的叫为“”的项目时.我多希望我不知道“”.我的大脑一片空白.我希望我不知道她说的,但其实不是这样.当时我什么都没想.

我不是在说我没必要学任何东西,相反地,我必须学习很多并且我犯了不计其数的错误。
但我宁愿避免情况或错误吗?我再也想不出不比这好的了.


All of us have the annoying tendency to look at the things that we could do
better, the things we do not know, and perceive them as weaknesses. But what
about weaknesses that are our strengths?

Here is my personal story about ignorance, naivety and false perception, and
about how happy I am I had no clue.

\section*{Names}

I had no idea who this guy was I met during the first day of my job. He entered
the room, introduced himself, and started asking tough questions that gave me
the impression that all I thought I would be doing was nonsense. He was
apparently well informed about my doings in KDE and the people I used to deal
with. Still we seemed to have different standpoints. At some point I grew tired
of his provocations and lost patience. I told him that things are not always as
easy with people as engineers wish they were.

It was only after he had left after about an hour of discussing that I googled
his name: Matthias Ettrich. What I read explained a lot about why he asked the
questions he did. If I had known before that he is one of the founders of the
KDE project I would have likely argued in a very different way -- if at all.

I had to look up quite some names during the last years, and I was happy every
single time that I did it \textit{after} the first contact.

This is probably my most important point. When I met all these FOSS people for
the first time I had almost never heard their names before. I did not know about
their history, their merits, nor their failures. I approached everyone in the
same way: on eye-level. 

By being ignorant (or naive, as some have called it), I did not feel inferior to
the people I met when I started my journey into FOSS land. I knew I had a lot to
learn but I never had the impression I had a lower position than others as a
person.

\section*{``High-Profile-Project''}

I had not religiously followed dot.kde.org nor PlanetKDE, let alone all those
countless other FOSS related publications before I started lurking on KDE
mailing-lists. I perceived those channels first and foremost as means of
communication to a very select audience, mainly users of and contributors to the
project itself. 

For quite some time, it did not even cross my mind that the articles I published
on The Dot might be picked up by journalists. I put an effort into writing them
because I wanted to do a good job rather than because I was afraid of making a
fool out of myself in the world's face. The press list was maintained by other
people and what I wrote did not appear that important to me either. I wanted to
reach certain people, and the official channels and my own blog seemed like the
most efficient way of doing it.

Being quoted on ReadWriteWeb after announcing on my blog that I would start a
new job almost came as a shock to me. It is not that I did not know that people
read what I write -- I certainly hope they do! -- I simply did not expect it to
be that much of a topic. It wasn't even summer break.

Good thing nobody told me; I would not have been able to publish a single line.

\section*{The Outsider}

Some time ago when I attended my first conference I did so with the firm belief
that I was different from the other attendees. I saw myself as an outsider
because I did not have much in common with anybody else apart from a fuzzy
interest in technology: I had been freelancing for some years already after
graduating from university, I had no relevant education in the field, and I was
mother of a 10 year-old child. On paper at least, it could not get much
different from the usual suspects one meets inside FOSS projects.

In 2008 I attended a KOffice sprint as part of the KDE marketing and promotion
team to prepare the 2.0 release. The initial idea was to sketch out a series of
promotional activities supporting the release to grow both developer and user
base, for which there were three of us running a parallel track to the developer
discussion.

We tried to understand how we could position KOffice and adapt communication to
the intended audience. Pretty soon in the process, we discovered that we had to
take a step back: at that point, the immaturity of the suite made it impossible
to position it as an option for unsuspecting users. We had to stick with
developers and early adopters. It was a tough sell to some of the developers but
as outsiders we had the chance to look at the software without thinking of all
the blood, sweat and tears that went into the code.

For a lot of projects, no matter of which kind they are, the core contributors
have a hard time taking an objective look at the state of affairs. We tend to
not see the great accomplishments while we are so focused on the issues in
detail, or the other way around. Sometimes we miss a good opportunity because we
\textit{think} it has nothing to do with what we are doing -- or that no-one
would want this in the first place.

In all these cases, people outside the project have the potential to inject some
different viewpoints into the discussion, particularly when it comes to
prioritization. It is even more helpful if they are not developers themselves:
they will ask different questions, will not feel pressured into knowing and
understanding all technical details, and they can help decisions and
communication on a higher level.

\section*{Conclusion}

Ignorance is bliss. It is not only true for the individuals who benefits from
the fearlessness that results from a lack of knowledge but also for the projects
these individuals join. They bring different views and experiences.

And now, go and find yourself a project that interests you, regardless of what
you think you know.
