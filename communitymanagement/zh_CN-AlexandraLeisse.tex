\chapterwithauthor{Alexandra Leisse}{那些我很庆幸曾经不知道的事}

\authorbio{Alexandra Leisse离开了一个舞台而来到了另一个,并把她对软件和网络的热情变成了职业。
在经历了十二个月的为软件和歌剧自由撰稿的过渡期,以及在KDE相关活动投入不计其数的时间后,
她加入了诺基亚的Qt开发框架团队,成为了一位社区管理员。

\newline
她是那位存在于Qt开发者网络和在线Qt社区活动幕后的女人。
尽管在歌剧表演上颇有成就,她几乎拒绝在公开场合歌唱。}

\section*{前言}

当Lydia邀请我加入她的题为“那些我希望当时就知道的事”的新书项目时,我的思绪一片空白。
那些我希望当时就知道的却不知道的事?我想不起来有这样的事。

我不是在说我不需要学习任何事,相反,我必需学习很多并且在过程中犯下了数不清的错误。
但是那其中有我期望避免的情况或错误吗?我认为没有。

我们所有人都有这样的烦人的趋向,就是在意那些我们可以做得更好和我们所不知的事,
并且认为这些是弱点。但当弱点成为我们的强项时又会如何?

以下是我个人的关于无知、天真和错觉的故事,也是我自己也不知道我有多么快乐的故事。

\section*{名字}

我记不清在我工作第一天遇到的那位同事是谁。
他进入了房间,自我介绍后开始问我各种难题,这给了一种那些我将要做的事情都是没有意义的印象。
他显然很清楚我将要在KDE社区从事的工作以及我之前打交道的人。
然而我们看起来有不同的立场。在一些观点上,我逐渐对他的挑衅感到厌烦并且失去耐心。
我告诉他和人打交道这类事情并不总是像工程师们认为的那样容易。

It was only after he had left after about an hour of discussing that I googled
his name: Matthias Ettrich. What I read explained a lot about why he asked the
questions he did. If I had known before that he is one of the founders of the
KDE project I would have likely argued in a very different way -- if at all.

I had to look up quite some names during the last years, and I was happy every
single time that I did it \textit{after} the first contact.

This is probably my most important point. When I met all these FOSS people for
the first time I had almost never heard their names before. I did not know about
their history, their merits, nor their failures. I approached everyone in the
same way: on eye-level. 

By being ignorant (or naive, as some have called it), I did not feel inferior to
the people I met when I started my journey into FOSS land. I knew I had a lot to
learn but I never had the impression I had a lower position than others as a
person.

\section*{``High-Profile-Project''}

I had not religiously followed dot.kde.org nor PlanetKDE, let alone all those
countless other FOSS related publications before I started lurking on KDE
mailing-lists. I perceived those channels first and foremost as means of
communication to a very select audience, mainly users of and contributors to the
project itself. 

For quite some time, it did not even cross my mind that the articles I published
on The Dot might be picked up by journalists. I put an effort into writing them
because I wanted to do a good job rather than because I was afraid of making a
fool out of myself in the world's face. The press list was maintained by other
people and what I wrote did not appear that important to me either. I wanted to
reach certain people, and the official channels and my own blog seemed like the
most efficient way of doing it.

Being quoted on ReadWriteWeb after announcing on my blog that I would start a
new job almost came as a shock to me. It is not that I did not know that people
read what I write -- I certainly hope they do! -- I simply did not expect it to
be that much of a topic. It wasn't even summer break.

Good thing nobody told me; I would not have been able to publish a single line.

\section*{The Outsider}

Some time ago when I attended my first conference I did so with the firm belief
that I was different from the other attendees. I saw myself as an outsider
because I did not have much in common with anybody else apart from a fuzzy
interest in technology: I had been freelancing for some years already after
graduating from university, I had no relevant education in the field, and I was
mother of a 10 year-old child. On paper at least, it could not get much
different from the usual suspects one meets inside FOSS projects.

In 2008 I attended a KOffice sprint as part of the KDE marketing and promotion
team to prepare the 2.0 release. The initial idea was to sketch out a series of
promotional activities supporting the release to grow both developer and user
base, for which there were three of us running a parallel track to the developer
discussion.

We tried to understand how we could position KOffice and adapt communication to
the intended audience. Pretty soon in the process, we discovered that we had to
take a step back: at that point, the immaturity of the suite made it impossible
to position it as an option for unsuspecting users. We had to stick with
developers and early adopters. It was a tough sell to some of the developers but
as outsiders we had the chance to look at the software without thinking of all
the blood, sweat and tears that went into the code.

For a lot of projects, no matter of which kind they are, the core contributors
have a hard time taking an objective look at the state of affairs. We tend to
not see the great accomplishments while we are so focused on the issues in
detail, or the other way around. Sometimes we miss a good opportunity because we
\textit{think} it has nothing to do with what we are doing -- or that no-one
would want this in the first place.

In all these cases, people outside the project have the potential to inject some
different viewpoints into the discussion, particularly when it comes to
prioritization. It is even more helpful if they are not developers themselves:
they will ask different questions, will not feel pressured into knowing and
understanding all technical details, and they can help decisions and
communication on a higher level.

\section*{Conclusion}

Ignorance is bliss. It is not only true for the individuals who benefits from
the fearlessness that results from a lack of knowledge but also for the projects
these individuals join. They bring different views and experiences.

And now, go and find yourself a project that interests you, regardless of what
you think you know.
