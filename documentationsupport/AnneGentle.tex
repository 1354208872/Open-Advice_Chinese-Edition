\chapterwithauthor{Anne Gentle}{Documentation and My Former Self}

\chapterwithauthor{Anne Gentle}{参考资料和我之前的人生}

\authorbio{Anne Gentle is the fanatical technical writer and community
documentation coordinator at Rackspace for OpenStack, an open source cloud
computing project. Prior to joining OpenStack, Anne worked as a community
publishing consultant, providing strategic direction for professional writers
who want to produce online content with wikis with user-generated pages and
comments. Her enthusiasm for community methods for documentation prompted her to
write a book about using social publishing techniques for technical
documentation titled Conversation and Community: The Social Web for
Documentation. She also volunteers as a documentation maintainer for FLOSS
Manuals, which provides open source documentation for open source projects.}

\authoribio{Anne Gentle 是狂热的技术文档撰稿人和OpenStack的Rackspace的社区文档协调员。比起加入OpenStack,
Anne更愿意做一个社区出版顾问,因为该职业能为想用用户生成页面的维基和评论来建立网上内容的专业作家提供战略性指导。
她的为社区方式的参考资料的热情促进她去写一本关于出版公共出版技术,其内容针对有技术文档标题的对话和社区:
The Social Web for Documentation(公共网络的文献),她也作为志愿者参加FLOSS Manual说的文献修复者,为开源项目
提供开源资源
}

\authorbio{Anne Gentle }

\authorbio{Anne Gentle}

\noindent{}An intriguing premise -- spill my guts about what I wish I knew about open
source and documentation. Rather than tell you what I wish you knew about open
source and documentation, I must tell you what I wish my former self knew. The
request evokes a sense of regret or remorse or even horrified notions of ``What
was I thinking?'' 

\noindent{}一个有趣的前提——泄露我希望你知道开源资源和文档的部分。而不是告诉你我希望你了解
源代码和文档,我必须告诉你我希望我自己之前能够知道。
这个唤醒了后悔和遗憾的感觉,甚至“我在想什么”的惊骇。


In my case, my former self was just five years younger than now, a
thirty-something established professional. In contrast, others may recall their
first experiences with open source as a teenager. Jono Bacon in his book,
\textit{Art of Community}, recounts standing in front of an apartment door with
his heart pounding, about to meet someone he had only talked to online through
an open source community. I have experienced that first in-person meeting with
people I have only met online, but my first serious foray into the world of open
source documentation came when I responded to an emailed request for help. The
email was from a former coworker, asking for documentation help on the XO
laptop, the charter project for One Laptop Per Child. I pondered the perceived
opportunity, talking to my friends and spouse, wondering if it would be a good
chance to experiment with new doc techniques and try something I had never done
before, wiki-based documentation. Since that first experimentation, I have joined
OpenStack, an open source project for cloud computing software, working full
time on community documentation and support efforts. 

I immediately think of the many contradictions I have found along the way. I
have uncovered surprising points and counterpoints for each observation. For
example, documentation absolutely matters for support, education, adoption, yet,
an open source community will march on despite a lack of documentation or
completely flawed docs. Another seeming juxtaposition of values is that
documentation should be a good onboarding task, a starting point for new
volunteers, yet new community members know so little that they can not possibly
write or even edit effectively, nor are newbies familiar with the various
audiences served by doc. Word on the street lately is that ``developers should
write developer docs'' because they know the audience well and can serve others
like themselves best. In my experience, new, fresh eyes are welcome to the
project and some people are able to write down and share with others those
fresh, empathetic eyes. You do not want to create a ``newbies-only'' culture
around docs, though, because it is important that the key technical community
members support doc efforts with their contributions and encourage others to do
so. 

A bit of a dirty little secret for documentation related to open source projects
is that the lines drawn between official docs and unofficial doc projects are
blurred at best. I wish I had known that documentation efforts get forked all
the time, and new web sites can sprout up where there were none. Sprawling docs
do not offer the most efficient way for people to learn about the project or
software, but a meandering walk through a large set of web documentation might
be more telling to those who want to read between the lines and interpret what
is going on in the community through the documentation. Lots of forking and
multiple audiences served may mean that the product is complex and serves many.
It also can mean that no strong core documentation ethos exists in the
community, so unorchestrated efforts are the norm. 

I wish when I started that I had some ability to gather the ``social weather''
of an online community. When you walk into a restaurant with white tablecloths
and couples dining and a low-level volume of conversations, the visual and
auditory information you receive sets the ambiance and gives you certain clues
about what to expect from your dining experience. You can translate this concept
of social weather to online communities as well. An open source community gives
certain clues in their mailing lists, for example. A list landing page prefixed
with a lot of rules and policy around posting will be heavy in governance. A
mailing list that has multiple posts emphasizing that ``there are no dumb
questions'' is more approachable for new doc writers. I also wish I knew of a
way to not only do a content audit -- a listing of the content available for the
open source project -- but also to do a community audit -- a listing of the
influential members in the open source community, be they contributors or
otherwise.

Lastly, an observation about open source and doc that I have enjoyed validating
is the concept that documentation can occur in ``sprints'' -- in short bursts of
energy with a focused audience and outline and resulting in a known set of
documentation. I was so happy to hear at a talk at SXSW Interactive that sprints
are perfectly acceptable for online collaboration and you could expect lags in
energy level, and that is okay. Software documentation is often fast and furious
in the winding-down-days of a release cycle, and that is acceptable in open
source, community-based documentation. You can be strategic and coordinated and
still offer a high-energy event around documentation. These are exciting times
in open source, and my former self felt it! It is a good thing you can keep
learning and growing your former self into your current self with the collection
of advice to tote along with you.
