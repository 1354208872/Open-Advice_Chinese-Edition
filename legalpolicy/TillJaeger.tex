\chapterwithauthor{Till Jaeger}{On being a Lawyer in FOSS} 

\authorbio{Dr. Till Jaeger has been a partner at JBB Rechtsanwaelte since 2001. He
Till Jaeger博士自从2001年来就成为了JBB Rechtsanwaelte的一员。
is a Certified Copyright and Media Law Attorney and advises large and
他既是版权和媒体法认证律师、顾问,运营着
medium-sized IT businesses as well as government authorities and software
一个中等规模的IT公司,也是政府机关和软件
developers on matters involving contracts, licensing and online use. One
涉及合同,许可和在线使用事项的开发者。
particular focus of his work is on the legal issues created by Free and Open
他的工作中的一个特别关注点是由自由
Source Software. He is co-founder of the Institute for Legal Aspects of Free \&
和开源软件引发的法律问题。他是自由和开源软件的法
Open Source Software (ifrOSS). He provides advice on compliance with open source licenses and on
律方面的学院联合创始人。他提供了开源证书的合规方
compatibility issues, and helps developers and software companies to enforce
面和兼容性方面的的建议,也帮助开发者和软件公司获得
licenses. Till represented the gpl-violations.org project in several lawsuits to enforce
许可。至今他代表了gpl-violations.org工程在好几次诉讼中执行GPL,
the GPL and has published several articles and books related to legal questions
也发表了一些关于自由和开源软件的
of Free and Open Source Software. He was a member of the Committee C in the
合法问题的文章和书。他是GPLv3拉力进程中C
GPLv3 drafting process.}
委员会的一个成员。
\noindent{}One thing upfront: I am not a geek. I never have been one, and have no
首先说一件事:我不是一个极客,我从未当过极客,也没有
intention of becoming one in the future. 
任何在未来成为极客的打算
Instead, I am a lawyer. Most people who read this book probably tend to
相反的,我是一个律师。读这本书的大多数人可能更多地去
sympathize more with geeks than with lawyers. Nevertheless, I do not want to
同情极客而不是律师。虽然如此,我并不想去
hide this fact. That the FOSS community is not necessarily fond of lawyers but
掩盖这个事实。我确实知道在1999年当我们第
busy developing software is something I \textit{did} know about FOSS in early
一次打交道时FOSS社区不
1999 when our ways first crossed. But there were also quite a few things I did
一定喜欢律师但却忙于开发软件。但是有很多事我确实
not know.
不知道
In 1999, while completing my doctoral thesis that focused on a classical
在1999年,当我正在完成我的有关经典版权的博士论文的时候,
copyright topic, I was assessing the scope of moral rights. In this context I
我正在评估道德权利的范围。在这片文章上
spent a while pondering about the question of how moral rights of programmers
我花了一段时间琢磨了一下程序员的道德权利
are safeguarded by the GPL, which allows others to modify their programs. This
如何受GPL保护的问题,这个问题会允许其他人修改这些程序员的程序。
is how I first got in contact with FOSS. At the time, ``free'' and ``open''
这就是我第一次和FOSS接触。在那时,"自由"和"开源"
certainly had different meanings, but the difference was not worth arguing
确实有不同的意思,但是其中的不同并不值得在
about in the world I was living in. However, since I was free to do what I was
我们的生活中争论。然而,自从我自由地去做我喜欢的事情,
interested in and open to investigate new copyright questions, I soon found out
开放地去研究新的版权问题时,不久后我便发现
that the two words \textit{do} have something in common, that they are
这两个词在一些方面是相同的,
\textit{different} and yet they are best used together...
这两个词最好一起使用
There are three things I wish I had known back then:
有三件事情我希望我当时已经知道了:
First, my technical knowledge, particularly in the field of software, was
首先,我的专业知识,尤其是在软件领域,是不够充足的。
insufficient. Second, I did not really know the community and what mattered to
其次,我真的不清楚这个社区,也不知道
the people who were part of it. Last but not least, I did not know much about
里面的人关心什么。最后但并非最不重要,我并不知道很多
foreign jurisdictions back then. It would have been useful to know all that from
当时的外国的司法权。如果从一开始我就知道这些的话真的
the beginning.
会非常有用
Since that time, I have learned a fair bit, and just as the community is happy

to share its achievements I am happy to share my lessons\footnote{The ``Institut

f\"ur Rechtsfragen der Freien und Open Source Software'' (Institute for Legal

Questions on Free and Open Source Software) offers, inter alia, a collection of

FOSS related literature and court decisions; see www.ifross.org for details.}:

\paragraph*{Technical knowledge}
How is software architecture shaped? What is the technical structure of software
like? Which licenses are compatible with each other and which are not, and how
and why? How is the Linux kernel structured? 

To name one example, the important question of what constitutes a ``derivative
work'' according to the GPL determines how the software may be licensed.
Everything that counts as derived from GPL-licensed software must be distributed
under the GPL. To assess whether a certain software is a ``derivative work'' or
not requires profound technical understanding. The interaction of program
modules, linking, IPC, plugins, framework technology, header files and so on
determines, among other criteria, whether a program is formally inseparable,
which helps to determine whether it is derived from another program or not. 

\paragraph*{Knowledge of the industry and the community}
Besides these functionality issues I had no profound understanding of the idea
behind FOSS and the motivation of the developers and the companies that use
FOSS. Neither did I really know about its philosophical background, nor was I
familiar with practical issues such as ``who is a maintainer?'' or ``how do version control systems work?'' In order to serve your clients best, these matters are no less important than your proficiency in technical aspects.  
Our clients ask us about legal aspects of forming business models such as dual
licensing, ``open core'', support and services contracts, code development and
code contribution agreements. We consult clients concerning what FOSS might have
in store for their companies or institutions. We also advise developers on what
they can do about infringement of their copyrights, and draft and negotiate
contracts for them. In order to serve such clients comprehensively, it is
important to be familiar with the different points of view.  

\paragraph*{Comparative law knowledge}
The third thing a FOSS lawyer needs is knowledge about foreign jurisdictions, at
least a few, and the more the better. In order to construe the different
licenses, it is essential to be familiar with the perspective of the people who
have drafted it. In most cases the U.S. legal system is of key importance. For
example, the GPL was drafted with U.S. legal concepts in mind. In the United
States, ``distribution'' includes online distribution, whereas under the German
Copyright system there is a distinction between offline and online distribution.
Licenses that have been drafted by lawyers from the United States may thus be
construed as including online distribution, which might be relevant and helpful
in court proceedings\footnote{\url{http://www.ifross.org/Fremdartikel/LGMuenchenUrteil.pdf}, Cf. Welte v. Skype, 2007}.
 
\section*{Always Learning}
So, all this is useful to know. And as software keeps on being developed and
modified to provide solutions for the needs of the day, so my mind will
hopefully keep on finding answers to the challenges the vibrant FOSS community
poses to a lawyer's mind.
