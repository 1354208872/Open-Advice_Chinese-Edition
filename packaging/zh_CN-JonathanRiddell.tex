\chapterwithauthor{Jonathan Riddell}{From Beginner to Professional}从菜鸟到专家

\authorbio{Jonathan Riddell is a KDE and Kubuntu developer currently employed by
Canonical. When not at a computer he canoes the rivers of Scotland.}
Jonathan Riddell是KDE和Kubuntu的开发者,目前供职于Canonical。
他不是在苏格兰溪流上划艇就是在计算机前面。

\noindent{}There was a bug in the code. A nasty one too: a crash without saving data.
That is the problem with looking at code, you find things to fix. It is easy to get
involved in Free Software; the difficult part is getting out again. After the
first bug fix there are more and more, all within reach. Bug fixes lead to
adding features, which leads to project maintenance, which leads to running
community. 
在代码上有严重的漏洞。让人不堪的是没有保存数据而程序崩溃。
那是你看到代码有问题,想方设法去修复。
接触自由软件是简单的,困难的部分在于一直坚持。
当第一个代码修复之后有越来越多的漏洞涌现
。修复代码导致添加附加特性,使项目维持和社区运行。

It started with reading Slashdot, that mass of poorly filtered tech and geek
news with comments from anyone who can reload fast enough to get at the top.
Every news story was interesting and exciting, a fresh insight into the tech
world I was becoming fascinated with. No more did I have to accept what was
given to me by large software companies, here in the Free Software community I
could see the code develop in front of me.
从阅读Slashdot这个资讯科技网站开始,
大量被劣质过滤的科技和极客新闻通过那些重复加载网页来评论的人冲到的头条。
每一个新闻故事都是有趣并使人激动的,我着迷那些在科技世界里产生的新鲜的深刻的见解。
我不再必须接受被那些大型软件公司灌输的东西,在这个自由软件社区我能够亲眼见证代码的发展。

As a university student it was possible to complete the exercises given by
lecturers very quickly, but exercises are not finished programs. I wanted to
know how to apply the simple skills they had given me to the real world by
writing programs which solve real problems for people. So I looked for the code,
which was not hard to find, just lying around on the Internet in fact. 
Looking closer at the code for the programs I was running I saw beauty. Not
because the code was perfectly tidy or well-structured, but because I could
understand it with the concepts I had already learned. Those classes, methods
and variables fell into place, enabling me to solve the relevant problems. Free
Software is the best way to make that step from knowing how to finish exercises
in a class to understanding how real programs get written.
作为大学的学生是可能快速完成讲义上提供的练习的,但是完成练习不意味着完成了程序。
我想要去了解如何应用他们提供的简单的技巧来写可以为人们解决现实世界的问题的程序。
因此我寻找那些不是太难找代码,事实上,这些代码就存储在因特网中。
这些类、方法和变量依序排列,是我能够解决相关的问题。
自由软件是最好的方式来一步一步的知道如何完成在一堂课里的练习,
以此来理解一个真正的程序是如何被写成的。

Every computing student should work on Free Software for their dissertation.
Otherwise you get to spend six months to a year on a project only for it to sit
in the basement of a library never to be visited again. Only Free Software makes
it possible to excel by doing what comes naturally: wanting to learn how to
solve interesting problems. By the end of my project NASA programmers were using
my UML diagramming tool and it won awards with lavish receptions. With Free
Software you can solve real problems for real users.
每一个计算机专业的学生应该通过致力于自由软件来完成一篇学位论文。
否则你花了六个月到一年的时间仅仅使用了基本的库却从没访问过这些。
也只有自由软件使出色得完成来得自然变得可能:想要学习如何解决感兴趣的问题。
在最后我的项目NASA程序员使用了我的UML图表工具使我获得了大量名誉。
通过自由软件你可以为真正的用户解决真正的问题。

The developer community is full of amazing people, with the passion and
dedication to work without any more reward than a successful computer program.
The user community is also awesome. It is satisfying to know you have helped
someone solve a problem, and I appreciate the thank you emails I receive.
程序员社区里有大量令人惊奇的人,他们拥有热情和不要任何报酬的奉献精神为去工作为了一个成功的计算机软件。
这个用户社区也非常令人敬畏。你帮助某个人解决问题是令自己满足的,并且我感激我收到感谢信。

Having written useful software, it needs to be made available to the masses.
Source code is not going to work for most people, it needs to be packaged up.
Before I was involved in it I looked down on packaging as a lazy way to
contribute to Free Software. You get to take much of the credit without having
to code anything. This is somewhat unfair, much of the community management
needed to run any Free Software project can also be seen as taking the credit
without doing the code.
写成一个有用的软件需要聚集。
源代码不是为了大部分人存在的,这些需要被打包。
在我涉及这个之前我轻视了封装对于自由软件的贡献,只把这个当做一个懒惰的方法。
你开始花大量时间在不需要编程的信用级别上。
这稍微有点不公平,许多社区管理也能够看到需要运行任一一个自由软件项目,
因为获得了信用等级不需要参与写代码。

Users depend on packagers a lot. It needs to be both fast, to keep those who
want the latest and greatest, and it needs to be reliable, for those who want
stability (which is everyone). The tricky part is that it involves working with
other people’s software, and other people’s software is always broken. Once
software is out in the wild problems start to emerge that were not visible on
the author’s own computer. Maybe the code does not compile with a different
compiler version, maybe the licensing is unclear so it can not be copied, maybe
the versioning is inconsistent so minor updates might be incompatible, screen
sizes might be different, desktop environments can affect it, sometimes
necessary third party libraries do not even have a release. These days software
needs to run on different architectures, 64-bit processors caused problems when
they became widely available, these days it is ARM which is defeating coders’
assumptions. Packagers need to deal with all of these issues, to give something
to the users which reliably works.
用户非常依赖于打包员。它需要快,为了拥有那些
希望最新和最好的特性的人们,它需要可靠,对于那些想要
稳特性的人(对于每个人)。棘手的部分是,它涉及到使用
别人的软件,然而别人的软常常出现问无法工作。一旦
软件开始出现超出常见的自然的问题就那些不可见的问题在
作者自己的电.可能由于用不同的编译器版本进行编译,
还有许可证不够明确以至于不能(被别人)复制,也许
版本不一致,以至于小更新可能导致不兼容,也有可能屏幕
大小可能不同,桌面环境也能够影响它,有时
必要的第三方库甚至没有释放。这些天软件
需要不同的体系结构上运行,当软件普及时64位处理器可能就会出现问题,
这些天正是ARM否定了程序员的
假设。打包员需要处理所有这些问题,付出一些东西为了
让用户得到可靠地工作。

We have a policy in Ubuntu that packages with unit tests must have those tests
enabled as part of the package build process. Very often they fail and we get
told by the software author that the tests are only for his or her own use.
Unfortunately it is never reliable enough in software to test it yourself, it
needs others to test it too. One test is rarely enough, it needs a multi-layered
approach. The unit tests from the original program should be the first place to
start, then the packager tests it on his or her own computer, then it needs
others to test it too. Automatic install and upgrade testing can be scripted on
cloud computing services quite nicely. Putting it into the development
distribution archive gets wider testing before finally some months later it gets
released to the masses. At each stage problems can and will be found which need
to be fixed, then those fixes need testing. So there might not be much coding
involved but there is a lot of work to get the software from being 95\% to being
100\% ready, that 5\% is the hardest part, a slow and delicate process needing
careful attention all the way.
我们有一个政策在Ubuntu这就是包必须通过这些单元测试使包
成为构建过程的一部分。他们经常失败,从而我们从能够软件作者哪里
得到启示就是测试只供自己使用。
不幸的是你自己无法用软件以达到足够的可靠性地去测试它,
它需要别人来测试它。一个测试是不够的,它需要一个多层次的
的方法。最初程序的单元测试应该从最初位置
开始,然后打包员测它在他或她自己的电脑上,同样它需要
其他人也来测试它。自动安装和升级测试可以仿照
云计算服务。最后几个月投入开发分发存档获得更广泛的测试,
然后公布给大众。在每个阶段那些能够或者可能被发现的需要被修复的问题
然而那些补丁同样需要测试。所以可能不要太多编码需要的是许多的工作
为了让软件从95%完成度到100%的完成度,但是5%是最难的部分,一个缓慢而复杂的过程需要
一直小心。

You can not do packaging without good communication with your upstream
developers. When bugs happen it is vital to be able to find the right person to
talk to quickly. It is important to get to know them well as friends and
colleagues. Conferences are vital for this as meeting someone gives much more
context to a mailing list post than a year of emails can. 

One of the unspoken parts of the Free Software world is the secret IRC channels
used by core members of a project. All big projects have them, somewhere out
there Linus Torvalds has a way of chatting to Andrew Morton et al about what is
good and what is bad in Linux. They are more social than technical and when
overused can be very anti-social for the community at large, but for those times
when there is a need for a quick communication channel without noise they work
well.

Blogging is another important method of communication in the Free Software
community. It is our main method of marketing and promotion for both the
software we produce and ourselves. Not to be used for shameless self-publicity,
there is no point claiming you will save lives with your blog, but used to talk
about your work on Free Software it builds community. It can even get you a job
or recognized in the street.

Those Slashdot stories of new technology developments are not about remote
figures you never meet in the way newspaper stories are. They are about people
who found a problem and solved it using the computer in front of them. For a few
years I was editing the KDE news site, finding the people who were solving
problems, creating novel ideas and doing the slow slog of getting the software
up to high enough quality, then telling the world about them. There were never a
shortage of people and stories to tell the world about. 

My last piece of advise is to stay varied. There is such a wealth of interesting
projects out there to explore, learn from and grow, but once in a position of
responsibility it can be tempting to stay there. Having helped create a
community for Kubuntu I am moving temporarily to work on Bazaar, a very
different project with a focus on developers rather than non-tech users. I can
start again learning how code turns into useful reality, how a community
interacts, how quality is maintained. It will be a fun challenge and I am
looking forward to it.
