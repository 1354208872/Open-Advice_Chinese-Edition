\chapterwithauthor{Runa Bhattacharjee}{My Project Taught Me how to Grow Up}

\authorbio{For the past 10 years, Runa Bhattacharjee has been translating and working on
localizing numerous Open Source projects - ranging from Desktop interfaces to
Operating System tools and lots of things in between. She strongly believes that
upstream repositories are the best places to submit any form of changes. She also
holds a professional portfolio specializing in Localization, at Red Hat. Runa
translates and maintains translations for Bengali (Indian version), but is
always happy to help anyone wanting to get started with localization.
在过去的十年里,Runa Bhattacharjee已经翻译和参与了许多开源项目——从桌面界面到操作系统工具
以及跟两者相关的很多东西。她坚信上游库是最好的地方来提交任何形式的改变。她还建立了一个专门
的文件夹。Runaway还翻译和维护孟加拉语的译文,但是她很高兴去帮助那些相邀开始局部化工作的人。}

\section*{Introduction}

Burning the midnight oil has been a favorite form of rebellion by young people
all over the world. Whether to read a book with a torchlight under the covers or
to watch late night TV reruns or (amongst other things) to hang around on an IRC
channel and tinkering around with an itchy problem with a favorite open source
project. 
  在全世界范围内,熬夜已经成为了年轻人反抗的一种方式。不论是躲在被子底下用手电筒
  照明看书,看午夜节目的重播,在一个聊天室里徘徊还是胡乱修改一个著名开源项目中的问题。
\section*{How it all began}

That is how it all began for me. Let me first write a bit about myself. When I
got introduced to the local Linux Users Group in my city, I was in between jobs
and studying for a masters degree. Very soon I was a contributor to a few
localization projects and started translating (mostly) desktop interfaces. We
used a few customized editors with integrated writing methods and fonts. The
rendering engines had not matured well enough to display the script with zero
errors on the interfaces, nonetheless we kept on translating. My focus was on
the workflow that I had created for myself. I used to get the translatable
content from the folks who knew how things work, translate it as best as I
could, add in the comments to help the reviewers understand how I comprehended
the text, filled in the necessary information for copyright and credits and sent
them back to the coordinators.
对我来说,这一切就是这样开始的。首先让我先写一点我自己的事。当我被介绍给我所在城市
当地Linux用户组的时候,我被夹在了工作和获得硕士学位两者之间。很快,我就成为了一些
局部化项目的贡献者,开始翻译桌面界面。我们采用了一些带有互相协调协作方式和字体的编辑器。
渲染引擎还没有成熟到能够零错误地在界面中展示脚本内容,尽管我们在不断地做着翻译工作。
我关注的主要是我为自己制定的工作流程。过去我常常从哪些知道事情是怎样工作的人们那里拿来可译的内容,
尽我所能去把它们翻译好,添加到注解里来帮助评审员理解我是怎样组合文本的,填上版权和参与人员等必要信息,发给调整的人。
