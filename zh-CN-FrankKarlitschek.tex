\section*{The need for an ecosystem}
These are the main reasons why I want to see Free Software, and especially the
这些是为什么我想看到软件自由的主要理由,尤其是
free desktop, become mainstream. To make this happen, we need a lot more
免费桌面成为了主流。为了促使这件事,我们需要比今天更多的
contributors than we have today. By contributors I mean people who write the
贡献者。我所说的贡献者是指那些写
core frameworks, the desktop, the great applications. We need people who work on
核心框架,桌面,伟大的应用的人。我们需要从事于
usability, artwork, promotion and many other important areas. KDE is already a
适用性,美工,推销以及许多其他重要领域的人员。KDE已经是一个
really big community with thousands of members. But we need more people to help 
相当大的团体,有上千成员。但在与专利软件竞争这条艰难的路上,
to compete with proprietary software in a big way. The Free Software community
我们需要更多的人的帮助。在世界上,自由软件团体
is tiny compared to the proprietary software world. On the one hand this is not
相对于专利软件是弱小的。从一方面来看,这并不是
a problem, because the distributed software development model of the Free
一个问题,因为世界上,自由软件的分布式软件发展模型
Software world is much more efficient than the closed source way of writing
比写软件的闭源方式有效率多了。
software. One big advantage is, for example, the ability to re-use code better.
比如说,一个很大的好处是重复使用的代码的质量会越来越好。
But even with these advantages we need many more contributors than we have
但即使有这些好处,如果我们真的想要攻克桌面和手机市场,
today, if we really want to conquer the desktop and mobile markets.
我们仍需要比今天更多的贡献者。
We also need companies to help us bring our work to the mass market. In a
我们也需要公司的帮助,将我们的成果带进大规模的市场。
nutshell, we need a big and healthy ecosystem that enables people to work on
简单的说,我们需要一个巨大的,健康的社会环境使人们能够
Free Software for a living.
以从事自由软件为生
\section*{The current situation}
