\section*{前言}

这是一本讲述社区与技术的书。这本书
展现了集体的努力,这更像是我们一起建立的技术。
如果这是你初次与我们的社区相遇,
你也许会觉得由一个社区来推动技术发展,这很奇怪。
技术不应该是由大型公司所建立的吗?
事实上,对我们来说,这正好相反。

这本书的作者们全是自由软件社群——你可以这样称呼——的成员们。
在这个社群中,人们分享着这样一些基本的
经验:软件应该是更自由、更有用、更
灵活、更可控、更合理、包容性更强、更加
可持续、更高效、更安全的,以及
当它伴随着4个基本的自由——可以使用、研究、分享和
改进软件——而来的时候,从根本上来说,这样会更好。

现在有越来越多的社群已经可以做到
不顾地理上的隔阂,依靠网络进行交流。
正是这个自由软件社群开辟了这个新时代。

事实上,因特网和自由软件社群\footnote{对我来说,
  开源是那个社群的一个方面。在1998年,因特网产生之后
  的很短的时间内,这个特别的方面使它很好地表达了
  自己。但是如果开源是你更喜欢的术语的话,
  你可以用开源代替自由软件,这随便你。}是
相互依赖地发展着的。随着因特网的成长,我们的社群可以
与它一起成长,但是如果是去了我们社群的价值和技术的话,
毫无疑问因特网是不会成为现在我们所看到的这样一个能使
全世界的人们和群体相互交流的无所不包的网络的。

直到今天,我们的软件运行着大多数因特网,你也许
至少知道一些,比如Mozilla Firefox,
OpenOffice.org,Linux,也许甚至GNOME或KDE。但是我们的
技术或许也藏在你的电视中、无线
路由器中、自动取款机中,甚至是收音机、安全系统或战舰之中。
这样看来,它无所不在。

它在那些大公司的成功中起着重要的作用。
你知道的,比如Google,Facebook,Twitter以及其他公司。
如果不是软件自由的力量允许他们站在前人的肩膀上,
他们没有一个能在这么短的时间内获得如此巨大的成功。

但是也有许多小公司因自由软件而生、伴自由软件而存、靠
自由软件而活,包括我的Kolab Systems。对所有公司来说,
积极地参与到社群中去已经在诚信和持续方面成为一个决胜
因素。这一点大公司也是如此,就像
Oracle接管了Sun Microsystems之后就已经不自觉地
表现出来了。